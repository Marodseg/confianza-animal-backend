\thispagestyle{empty}

\begin{center}
{\large\bfseries \myTitle}\\
\end{center}
\begin{center}
       \myName\\
\end{center}

\vspace{0.7cm}
\noindent{\textbf{Palabras clave}: API, endpoint, schema, test, backend, frontend, Firebase, token, despliegue}\\

\vspace{0.7cm}
\noindent{\textbf{Resumen}}
\\
\\
El abandono animal es un problema real, serio y cada vez más común en España. Se estima que alrededor de unos 300.000
perros y gatos son recogidos al año por protectoras en todo el territorio estatal. Actualmente, las empresas tecnológicas
prefieren no apostar por proyectos sin ánimo de lucro, por lo que protectoras y asociaciones de animales no cuentan con
recursos suficientes para desarrollar herramientas que les ayuden a gestionar su trabajo. En este contexto, este proyecto
tiene como objetivo implementar una API REST que permita a las organizaciones de protección animal gestionar sus datos,
así como la información de los animales que buscan ser adoptados, y una página web para llevar a cabo dichas gestiones.
Todo ello por medio de tecnologías modernas como Fast API, Firebase y Angular.
\cleardoublepage
\thispagestyle{empty}


\begin{center}
{\large\bfseries \myTitleEn}\\
\end{center}
\begin{center}
       \myName\\
\end{center}

\vspace{0.7cm}
\noindent{\textbf{Keywords}: API, endpoint, schema, test, backend, frontend, Firebase, token, despliegue}\\

\vspace{0.7cm}
\noindent{\textbf{Abstract}}
\\
\\
Animal abandonment is a real, serious and increasingly common problem in Spain. It is estimated that around 300,000 dogs
and cats are picked up each year by shelters throughout the country. Currently, technology companies prefer not to bet
on projects without profit, so shelters and animal associations do not have enough resources to develop tools that help
them manage their work. In this context, this project aims to implement a REST API that allows animal protection
organizations to manage their data, as well as the information of the animals that seek to be adopted and a web page
to carry out these management. All this through modern technologies such as Fast API, Firebase and Angular.

\cleardoublepage
\thispagestyle{empty}

\noindent\rule[-1ex]{\textwidth}{2pt}\\[4.5ex]

Yo, \textbf{Manuel Ángel Rodríguez Segura}, alumno de la titulación Ingeniería Informática de la \textbf{Escuela Técnica Superior
de Ingenierías Informática y de Telecomunicación de la Universidad de Granada}, con DNI 49627033W, autorizo la
ubicación de la siguiente copia de mi Trabajo Fin de Grado en la biblioteca del centro para que pueda ser
consultada por las personas que lo deseen.

\vspace{6cm}

\noindent Fdo: Manuel Ángel Rodríguez Segura

\vspace{2cm}

\begin{flushright}
Granada a 29 de junio de 2023.
\end{flushright}


\cleardoublepage
\thispagestyle{empty}

\noindent\rule[-1ex]{\textwidth}{2pt}\\[4.5ex]

Dº. \textbf{\myProf}, Profesor del \myDepartment de la \myUni.

\vspace{0.5cm}

\textbf{Informa:}

\vspace{0.5cm}

Que el presente trabajo, titulado \textit{\textbf{\myTitle}},
ha sido realizado bajo su supervisión por \textbf{\myName}, y autorizo la defensa de dicho trabajo ante el tribunal
que corresponda.

\vspace{0.5cm}

Y para que conste, expido y firmo el presente informe en \myLocation a \myTime.

\vspace{1cm}

\textbf{El director:}

\vspace{5cm}

\noindent \textbf{\myProf}

\chapter*{Agradecimientos}
\thispagestyle{empty}

       \vspace{1cm}

A mi familia, amigos y pareja por su apoyo emocional y motivacional durante todo el trabajo realizado. Su energía positiva
ha sido un gran impulso para no dejar de lado el proyecto. \\

Por otro lado, agradecer a mi tutor, profesores y compañeros de la Universidad de Granada por hacer todo más fácil y
llevadero. Me habéis ayudado a crecer tanto personal como profesionalmente. \\

Por último, me gustaría dedicar el proyecto a todos aquellos que trabajan de forma voluntaria y sin ánimo de lucro en el mundo animal. Gracias por
vuestro tiempo, esfuerzo, trabajo y dedicación. Sin vosotros, no sería posible la existencia de los refugios y protectoras que cada
vez son más necesarias y que tanto ayudan a los animales que más lo necesitan. \\

Espero que esta idea anime a más desarrolladores a colaborar y aportar su granito de arena para que los animales puedan encontrar
un hogar digno y consigan una vida mejor. \\

