\chapter{Estado del arte}\label{ch:estado-del-arte}

En este apartado se analizará el estado del arte en el sector animal, más concretamente en la adopción de mascotas.
Primeramente se hará un análisis de las principales aplicaciones web que ofrecen la posibilidad de realizar
este proceso de forma interna y posteriormente se hará una conclusión en la que se justifique la necesidad de realizar
este trabajo.

\section{Miwuki Pet Shelter}\label{sec:miwuki}

\href{https://petshelter.miwuki.com/}{Miwuki} es una aplicación para facilitar el proceso de adopción de mascotas que
cuenta con la colaboración de la prestigiosa fundación Affinity. Además de contar con una versión móvil, tiene una
versión web que será la analizada a mayor profundidad en esta sección. \\

La web consiste en una página principal en la que se muestran las mascotas disponibles para su adopción así como
las diferentes protectoras que hay registradas en la aplicación. Gracias a una búsqueda personalizada se puede
encontrar la mascota que más se ajuste a las necesidades del usuario como por ejemplo el tamaño, la raza, el sexo,
la zona geográfica en España, etc. Además cuenta con una funcionalidad \textit{"Busca tu match!"} que permite encontrar
mascotas que se adapten a las características y estilo de vida del usuario, el cual necesitará estar registrado
en la aplicación. \\

Otra funcionalidad que ofrece es la posibilidad de dar en adopción una mascota con un límite máximo de 3 casos por cuenta.
Para ello, el usuario deberá registrarse en la aplicación y rellenar un formulario con los datos de la mascota para
publicitar el anuncio en la página principal. Además especifica que queda totalmente prohibida la venta y en caso de
detectar alguna publicación de este tipo, se procederá a su eliminación así como a la suspensión de la cuenta del
usuario. \\

Por último, podemos encontrar un apartado \textit{"Bolsa de pienso"} que redirige a una página que permite
realizar donaciones las cuales destinan el 90\% de su valor a las alimentos y el resto a los gastos de la
aplicación. \\

En la siguiente tabla se reflejan las características principales, ventajas y desventajas de la aplicación web
para tener una visión general de la misma: \\

\begin{table}[h]
\centering
\begin{tabular}{|p{6cm}|p{4cm}|p{4cm}|}
\hline
\textbf{Características principales} & \textbf{Pros} & \textbf{Contras} \\
\hline
Plataforma dedicada a la adopción de mascotas. &
Amplia variedad de animales para adoptar. &
Limitado a la adopción de mascotas. \\
\hline
Posibilidad de difundir la adopción de mascotas. &
Posibilidad de añadir a favoritos los animales que más te gusten. &
No existe un registro específico para las organizaciones. \\
\hline
Posibilidad de realizar donaciones. &
Posibilidad de dar en adopción una mascota. &
No existe un buscador de organizaciones. \\
\hline
\end{tabular}
\caption{Resumen de Miwuki Pet Shelter}
\label{tab:miwuki}
\end{table}


\section{Kerubi}\label{sec:kerubi}

\href{https://kerubi.es}{Kerubi} es una aplicación web para la adopción de mascotas. En su página principal se muestra
un buscador para encontrar el tipo de mascota que se desea adoptar (perros, gatos, conejos\ldots) y un seleccionador de
provincia. En la misma página en la parte inferior podemos ver los últimos animales que se han añadido a la aplicación
y tener la posibilidad de añadirlos a favoritos. \\

Los usuarios registrados tendrán la posibilidad de difundir la adopción de sus mascotas en la página principal por medio
de un formulario que deberá rellenar con los datos de la mascota. Además, los usuarios podrán realizar donaciones
para ayudar a las protectoras. \\

Por último, existe un registro específico para las organizaciones que deseen colaborar con la aplicación. En este caso,
las organizaciones deberán rellenar un formulario con los datos de la protectora y una vez aprobada, podrán publicar
los animales que deseen publicar en la web. \\

En la siguiente tabla se reflejan las características principales, ventajas y desventajas de la aplicación web
para tener una visión general de la misma: \\

\begin{table}[h]
\centering
\begin{tabular}{|p{6cm}|p{4cm}|p{4cm}|}
\hline
\textbf{Características principales} & \textbf{Pros} & \textbf{Contras} \\
\hline
Plataforma de adopción y difusión de mascotas. &
Amplia variedad de animales para adoptar. &
Algunas funcionalidades requieren registro. \\
\hline
Contacto con criadores y particulares. &
Opciones de adopción y difusión de mascotas. &
Falta de control en la gestión de adopción. \\
\hline
Amplio catálogo de animales. &
Posibilidad de encontrar razas específicas. &
Costos adicionales asociados a la adopción. \\
\hline
Funcionalidad de búsqueda avanzada con filtros personalizables. &
Interfaz intuitiva y fácil de usar. &
Exceso de publicidad en la plataforma. \\
\hline
\end{tabular}
\caption{Resumen de Kerubi}
\label{tab:kerubi}
\end{table}

\section{Kiwoko adopta}\label{sec:kiwoko}

\href{https://kiwokoadopta.org}{Kiwoko} es una aplicación web para la adopción de mascotas. En su página principal se muestra
una lista de enlaces con búsquedas rápidas como "Perros en Granada" o "Roedores en Albacete", un buscador para encontrar
el tipo de mascota que se desea adoptar (perros, gatos o roedores) y un seleccionador de tamaño, edad y provincia.
También se ven algunos resultados por defectos que van cambiando según se vayan realizando búsquedas. \\

Kiwoko se encarga de verificar que las organizaciones que se dan de alta en la plataforma cumplan con los requisitos
legales y consten como una protectora de animales oficial. Una vez verificadas, podrán publicar los animales que
deseen en la web. \\

En esta ocasión no contamos con una funcionalidad para que los usuarios puedan difundir la adopción de sus mascotas,
ni tampoco con la opción de añadir a favoritos los animales por los que más estés interesado. Esto permite
que los anuncios que aparecen en la web sean de organizaciones y no de particulares y cuenten con la verificación
de la aplicación. \\

En la siguiente tabla se reflejan las características principales, ventajas y desventajas de la aplicación web
para tener una visión general de la misma: \\

\begin{table}[h]
\centering
\begin{tabular}{|p{6cm}|p{4cm}|p{4cm}|}
\hline
\textbf{Características principales} & \textbf{Pros} & \textbf{Contras} \\
\hline
Plataforma dedicada a la adopción de mascotas. &
Amplia variedad de animales para adoptar. &
Funcionalidades limitadas. \\
\hline
Gran cantidad de detalles de cada animal en adopción. &
Alta verificación de las organizaciones. &
Interfaz con un diseño poco atractivo. \\
\hline
Posibilidad de búsqueda por ubicación y categoría de animales. &
Facilidad para contactar directamente con las organizaciones. &
Falta de añadir a favoritos los animales que más te gusten. \\
\hline
Enlace a la web de la organización para más información. &
Registro de organizaciones verificado. &
Falta de información actualizada sobre animales disponibles. \\
\hline
\end{tabular}
\caption{Resumen de Kiwoko Adopta}
\label{tab:kiwoko-adopta}
\end{table}


\section{Mascostas adopción}\label{sec:mascostas}

\href{https://mascotasadopcion.com}{Mascostas adopción} es una aplicación web sencilla para la adopción de mascotas.
A diferencia de las ya mencionadas anteriormente, esta aplicación no requiere de un registro para poder adoptar una
mascota. En su página principal se muestran las mascotas disponibles para su adopción y un buscador de provincia,
animal y sexo. Si estás interesado por un animal, puedes rellenar un formulario con datos personales y de contacto
para que la protectora se ponga en contacto contigo y avanzar en el proceso de adopción. \\

En caso de estar registrado como usuario, puedes añadir a favoritos los animales que más te gusten y así poder
tener un acceso más rápido a ellos. Además, puedes realizar donaciones para ayudar a las protectoras, ver los animales
que ya han sido adoptados o difundir la adopción de tus mascotas. \\

En esta aplicación no contamos con el rol de organización, por lo que los anuncios que aparecen en la web son
de particulares y no se puede garantizar que todos los animales que se publican sean reales o que cumplan con las
condiciones legales para su adopción. \\

En la siguiente tabla se reflejan las características principales, ventajas y desventajas de la aplicación web
para tener una visión general de la misma: \\

\begin{table}[h]
\centering
\begin{tabular}{|p{6cm}|p{4cm}|p{4cm}|}
\hline
\textbf{Características principales} & \textbf{Pros} & \textbf{Contras} \\
\hline
Plataforma dedicada a la adopción de mascotas. &
Amplia variedad de animales para adoptar. &
Falta de control en el proceso de adopción. \\
\hline
Posibilidad de búsqueda por ubicación y categoría de animales. &
Interfaz intuitiva, moderna y fácil de usar. &
Costos adicionales asociados a la adopción. \\
\hline
Posibilidad de difundir la adopción de mascotas. &
Posibilidad de realizar donaciones. &
Falta de filtrado de organizaciones. \\
\hline
Posibilidad de añadir a favoritos los animales que más te gusten. &
Posibilidad de ver los animales que ya han sido adoptados. &
Falta de información actualizada sobre animales disponibles. \\
\hline
\end{tabular}
\caption{Resumen de Mascostas en adopción}
\label{tab:mascotas-adopcion}
\end{table}

\section{Webs de protectoras y ayuntamientos}\label{sec:protectoras-ayuntamientos}

Además de las aplicaciones web que hemos analizado anteriormente, existen otras webs que ofrecen la posibilidad
de realizar la adopción de mascotas por medio de formularios en vez de en la propia aplicación. Estas webs
suelen ser de protectoras de animales o de ayuntamientos y suelen tener un diseño más sencillo que las aplicaciones
mencionadas anteriormente. Su objetivo es fomentar y difundir la adopción de animales pero debido a la falta de
interacción con los usuarios y de los bajos recursos que tienen, no suelen ser muy utilizadas. \\

Ejemplo de estas webs son:

\begin{itemize}
    \item \href{https://www.lasrozas.es/sanidad-y-servicios-sociales/adoptar-animales}{Ayuntamiento de Las Rozas}
    \item \href{https://csmpa.palma.cat/portal/PALMA/csmpa/contenedor1.jsp?seccion=s_fdes_d4_v1.jsp&codbusqueda=1720&language=es&codResi=1&layout=contenedor1.jsp&codAdirecto=981}{Ajuntament de Palma}
    \item \href{https://www.valladolid.es/es/programa-adopta}{Ayuntamiento de Valladolid}
    \item \href{https://protectorabcn.es/adopta/}{Protectora de Barcelona}
    \item \href{https://elrefugio.org/adopta/}{El Refugio}
    \item \href{https://nuevavida-adopciones.org/}{Nueva Vida Adopciones}
    \item \href{https://www.anaaweb.org/}{Asociación Nacional de Amigos de los Animales}
\end{itemize}

En la siguiente tabla se reflejan las características principales, ventajas y desventajas de este tipo de webs
para tener una visión general de las mismas: \\

\begin{table}[h]
\centering
\begin{tabular}{|p{6cm}|p{4cm}|p{4cm}|}
\hline
\textbf{Características principales} & \textbf{Pros} & \textbf{Contras} \\
\hline
Plataformas informativas sobre la adopción de mascotas. &
Amplia variedad de información de animales en adopción. &
Falta de posibilidad de adopción. \\
\hline
Posibilidad de búsqueda por ubicación y categoría de animales. &
Amplia variedad de información sobre protectoras. &
Interfaces poco atractivas y poco intuitivas. \\
\hline
Posibilidad de difundir la adopción de mascotas. &
Posibilidad de realizar donaciones. &
Falta de filtrado de organizaciones. \\
\hline
Toda la información se encuentra actualizada. &
Posibilidad de ver los animales que ya han sido adoptados. &
Fallos de seguridad. \\
\hline
\end{tabular}
\caption{Resumen de webs de protectoras y ayuntamientos}
\label{tab:protectoras-ayuntamientos}
\end{table}

\section{Conclusiones}\label{sec:conclusiones-estado-del-arte}

Una vez analizadas las principales aplicaciones web para la adopción de mascotas, podemos concluir que ninguna de las
mencionadas cuenta con una web privada para las organizaciones con el fin de ayudarlas a gestionar
toda la información con la que tratan y que se genera a diario. Además, ninguna de las aplicaciones permite la
gestión de las peticiones que se realizan a través de las mismas, por lo que las organizaciones no pueden mantener
una comunicación fluida con los usuarios que desean adoptar una mascota. También cabe destacar que algunas de las
webs no cuentan con un diseño atractivo y moderno, lo que puede provocar que los usuarios no se sientan atraídos
por la aplicación y no la utilicen sumando a esto la falta de información actualizada sobre los animales disponibles
para la adopción y en algunos casos la imposibilidad de realizar la adopción a través de la propia aplicación. \\

En este proyecto se pretende desarrollar una \textbf{API} que permita a las organizaciones gestionar toda la información
pertinente a su actividad y que les ayude a controlar todas las peticiones que se realizan a través de la aplicación, es
decir, crear un flujo de comunicación entre las organizaciones y los usuarios que deseen adoptar una mascota a través
de diferentes estados de la petición con información relevante para ambos y con mensajes que permitan mantener
una comunicación fluida y actualizada. \\

La página web que se desarrollará para el proyecto contará con una interfaz de usuario que permita a las organizaciones
un control claro y sencillo de toda la información y de las peticiones que se realicen a través de la aplicación móvil
que se encargará de desarrollar mi compañero de proyecto Alejandro y que haciendo uso de la \textbf{API}, implementará todas
las funcionalidades haciendo hincapié en mejorar los defectos que hemos encontrado en las aplicaciones analizadas
anteriormente. \\